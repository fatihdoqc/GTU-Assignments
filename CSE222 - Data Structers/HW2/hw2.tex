\documentclass{article}
\usepackage{graphicx}
\begin{document}
Q2)
\newline
\textbf{a)}
\newline
\paragraph{
Big-O notation means the worst case scenario which also means the maximum time that alghoritm can take.  So "The running time of algorithm A is at least $O(n^2)$" means "minimum of the maximum time that taken by alghoritm A is $n^2$. Big-O notation used for upper-bound but "at least" is a lower-bound term. That's why it is meaningles to say.
\newline
}
\textbf{b)}
\newline
Say f(n) = $n^2+n$ and $g(n) = n$,
\newline
max(f (n), g(n)) = f(n) so $ f(n) = \theta (n^2)$
\newline
$\theta (f(n) + g(n)) = \theta (n^2+n + n) = \theta (n^2)$. So it is true.
\newline
\newline
Say f(n) = $n$  and $g(n) = n^2+n$,
\newline
max(f (n), g(n)) = g(n) so $ g(n) = \theta (n^2)$
\newline
$\theta (f(n) + g(n)) = \theta (n^2+n + n) = \theta (n^2)$. So it is true.
\newline
\newline
Say f(n) = $n$ and $g(n) = n$
\newline
max(f (n), g(n)) = f(n) or g(n).Let's say f(n). So $ f(n) = \theta (n)$
\newline
$\theta (f(n) + g(n)) = \theta (n + n) = \theta (n)$. So it is still true.
\newline
\newline
And there is no other situation left. So its all above proves that max(f (n), g(n)) = $\theta$(f(n) + g(n))
\newline
\newline
\textbf{b)}
i) $\lim_{n \to \infty} \frac{2^{n+1}}{2^n} = 2$ 
\newline
So $\lim_{n \to \infty} \frac{2^{n+1}}{2^n}$ equals to a constant (c).
\newline
So $f(n) = \theta (g(n))$, ( g(n) is $2^n$ and f(n) is $2^{n+1}$ ).
\newline
So it is correct.
\newline
\newline
ii) $\lim_{n \to \infty} \frac{2^{2n}}{2^n} = 2^n = \infty$ 
\newline
So $f(n) = \Omega (g(n))$, ( f(n) is $2^{2n}$ and g(n) is $2^n$ ).
\newline
So it is false.
\newline
\newpage
iii)
\newline
\includegraphics[width=7cm, height=4cm]{x.png}
\newline
It is wrong, it must be $O(n^4)$ because we don't know about f(n). It could be quadratic or constant or something else.
\newline
\newline
Q3)
\newline
$3^n > $ is the greatest because it's base is 3 , the greatest number among others. Then,
\newline
If n is greater than or equal to 2 , $2^{n+1} > n2^n $ because it's base is lower than 3.Else, $n2^n > 2^{n+1} $ Then,
\newline
$ 5^{\log_2n} > 2^n $ because once in two $ 5^{\log_2n}$ acts like $5^n$ which is still greater than $2^n$. Then,
\newline
$n^{1.01} > \sqrt{n}$ because  $n^{1.01}$ is very close to being linear.Then,
\newline
If $ n > \log n$ , $ n log n > \log n^3$. Then,
\newline
log n is the smallest growing term. Because the logorithm grows the smallest.
\end{document}